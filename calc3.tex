% document setup
\documentclass[10pt]{article}                               % 10pt font size
\usepackage[margin=1in]{geometry}                           % set the margin to 1 inch
\usepackage{mathpazo}                                       % use the font Palatino

\renewcommand{\familydefault}{\sfdefault}                   % set our font to default

% other packages
\usepackage{amsmath, amssymb}
\usepackage{changepage}
\usepackage[most]{tcolorbox}
\tcbuselibrary{breakable}

% define colors
\definecolor{darkgreen}{RGB}{0,100,0}

% tcolorbox types
\newtcolorbox{defaultbox}{
    breakable,
    enhanced,
    colback=white,
    colframe=black,
    boxrule=0.5pt,
    arc=2mm,
    left=1em,
    right=1em,
    top=1em,
    bottom=1em,
    overlay first={\draw[line width=.5pt] (frame.south west)--(frame.south east);},
    overlay middle={\draw[line width=.5pt] (frame.south west)--(frame.south east);\draw[line width=.5pt] (frame.north west)--(frame.north east);},
    overlay last={\draw[line width=.5pt] (frame.north west)--(frame.north east);}
}
\newtcolorbox{notebox}{
    enhanced,
    colback=white,
    colframe=black,
    boxrule=0.5pt
}
\newtcolorbox{examplebox}{
    breakable,
    enhanced,
    boxrule=0.5pt,
    top=1em,
    bottom=1em
}

\begin{document}

% Title: Calculus 3 Study Guide
\begin{center}
    {\Huge \textbf{Calculus 3 Study Guide}} \\
    \vspace{0.5em}
\end{center}

%\vspace{1em}

% Section Title: Coordinate Systems
\section*{Coordinate Systems}


\begin{adjustwidth}{2em}{0pt}

    % Subsection Title: Polar Coordinates
    \subsection*{Polar Coordinates}

    \begin{adjustwidth}{2em}{0pt}

        \begin{notebox}

            \textbf{Note:} \\
            Polar coordinates describe a point in 2D space using two values:

            \begin{itemize}
                \item \( r \): the distance from the origin
                \item \( \theta \): the angle from the positive x-axis
            \end{itemize}

            This system is ideal for problems with circular or rotational symmetry, such as spirals or radial fields.

        \end{notebox}


        \vspace{0.5em}

        % Conversions
        \textbf{Polar to Cartesian}

        \begin{itemize}
            \item \( x = r\cos\theta \)
            \item \( y = r\sin\theta \)
        \end{itemize}

        \vspace{0.5em}

        \textbf{Cartesian to Polar}

        \begin{itemize}
            \item \( r = \sqrt{x^2 + y^2} \)
            \item \( \tan\theta = \frac{y}{x}  \rightarrow  \theta = \tan^{-1}\left(\frac{y}{x}\right) \)
        \end{itemize}

        \vspace{0.5em}


        \begin{examplebox}

            \subsection*{Converting Coordinates}

            \textbf{Example 1:} Convert the Cartesian point \( \left(3,3\right) \) into Polar coordinates.
            \vspace{0.5em}

            \begin{adjustwidth}{2em}{0pt}

                \[ r = \sqrt{x^2 + y^2} = \sqrt{3^2 + 3^2} = \sqrt{18} = 3\sqrt{2} \]

                \[ \theta = \tan^{-1}\left(\frac{y}{x}\right) = \tan^{-1}\left(\frac{3}{3}\right) = \tan^{-1}(1) = \frac{\pi}{4} \]

                \textbf{Answer:} \textcolor{darkgreen}{\( \left(3\sqrt{2}, \frac{\pi}{4}\right) \)}

            \end{adjustwidth}

            \vspace{1em}

            \textbf{Example 2:} Convert the Polar point \( \left(2, \frac{\pi}{3}\right) \) to Cartesian coordinates.
            \vspace{0.5em}

            \begin{adjustwidth}{2em}{0pt}

                \[ x = r\cos\theta = 2\cos\left(\frac{\pi}{3}\right) = 2 \cdot \frac{1}{2} = 1 \]

                \[ y = r\sin\theta = 2\sin\left(\frac{\pi}{3}\right) = 2 \cdot \frac{\sqrt{3}}{2} = \sqrt{3} \]

                \textbf{Answer:} \textcolor{darkgreen}{\( \left(1,\sqrt{3}\right) \)}

            \end{adjustwidth}

        \end{examplebox}

        \break

        \begin{examplebox}
        
            \subsection*{Converting Equations}

            \textbf{Example 1:} Convert the equation \( r^2 = 4 \) and \( \theta = \frac{\pi}{4} \) to Cartesian coordinates.
            \vspace{0.5em}

            \begin{adjustwidth}{2em}{0pt}

                \textbf{Step 1:} Polar to Cartesian conversions:

                \vspace{0.25em}

                \[ x = r\cos(\theta), y = r\sin(\theta) \]

                \vspace{0.5em}

                \textbf{Step 2:} Substitute the given values into these formulas:

                \vspace{0.25em}

                \[ x = 2\cos\left(\frac{\pi}{4}\right) = 2 \cdot \frac{\sqrt{2}}{2} = \sqrt{2} \]

                \[ y = 2\sin\left(\frac{\pi}{4}\right) = 2 \cdot \frac{\sqrt{2}}{2} = \sqrt{2} \]

                \vspace{0.5em}

                \textbf{Answer:}

                \vspace{0.25em}

                \[ x = \sqrt{2}, y = \sqrt{2} \]

                \vspace{0.5em}

                So the equation in Cartesian coordinates would be:
                \textcolor{darkgreen}{\( x^2 + y^2 = 4 \)}

            \end{adjustwidth}
            

        \end{examplebox}


    \end{adjustwidth}


    % Subsection Title: Cylindrical Coordinates
    \subsection*{Cylindrical Coordinates}

    \begin{adjustwidth}{2em}{0pt}

        \begin{notebox}

            \textbf{Note:} \\
            Polar coordinates describe a point in 3D space using three values:

            \begin{itemize}
                \item \( r \): The distance from the origin in the xy-plane
                \item \( \theta \): The angle from the positive x-axis in the xy-plane (same as polar coordinates)
                \item \( z \): The height above (or below) the xy-plane
            \end{itemize}

            This system is useful for objects with circular symmetry around the z-axis, like cylinders and spirals.

        \end{notebox}

        \vspace{0.5em}

        % Conversions
        \textbf{Cylindrical to Cartesian}

        \begin{itemize}
            \item \( x = r\cos\theta \)
            \item \( y = r\sin\theta \)
            \item \( z = z \)
        \end{itemize}

        \vspace{0.5em}

        \textbf{Cartesian to Cylindrical}

        \begin{itemize}
            \item \( r = \sqrt{x^2 + y^2} \)
            \item \( \tan\theta = \frac{y}{x}  \rightarrow  \theta = \tan^{-1}\left(\frac{y}{x}\right) \)
            \item \( z = z \)
        \end{itemize}

        \break

        \begin{examplebox}

            \subsection*{Converting Coordinates}
        
            \textbf{Example 1:} Convert the Cartesian point \( \left(3,3,4\right) \) into Cylindrical coordinates.
            \vspace{0.5em}

            \begin{adjustwidth}{2em}{0pt}

                \[ r = \sqrt{x^2 + y^2} = \sqrt{3^2 + 3^2} = \sqrt{18} = 3\sqrt{2} \]

                \[ \theta = \tan^{-1}\left(\frac{y}{x}\right) = \tan^{-1}\left(\frac{3}{3}\right) = \tan^{-1}(1) = \frac{\pi}{4} \]

                \[ z = 4 \]

                \textbf{Answer:} \textcolor{darkgreen}{\( \left(3\sqrt{2}, \frac{\pi}{4}, 4\right) \)}

            \end{adjustwidth}

            \vspace{1em}

            \textbf{Example 2:} Convert the Cylindrical point \( \left(2, \frac{\pi}{6}, 5\right) \) into Cartesian coordinates.
            \vspace{0.5em}

            \begin{adjustwidth}{2em}{0pt}

                \[ x = r\cos\theta = 2\cos\left(\frac{\pi}{6}\right) = 2 \cdot \frac{\sqrt{3}}{2} = \sqrt{3} \]

                \[ y = r\sin\theta = 2\sin\left(\frac{\pi}{6}\right) = 2 \cdot \frac{1}{2} = 1 \]

                \[ z = 5 \]

                \textbf{Answer:} \textcolor{darkgreen}{\( \left(\sqrt{3}, 1, 5\right) \)}

            \end{adjustwidth}

        \end{examplebox}

        \begin{examplebox}
        
            \subsection*{Converting Equations}
        
            \textbf{Example 1:} Convert the equation \( r = 3 \) to Cartesian coordinates.
            \vspace{0.5em}
        
            \begin{adjustwidth}{2em}{0pt}
        
                \textbf{Step 1:} Cylindrical to Cartesian conversion:
        
                \vspace{0.25em}
        
                \[ r = \sqrt{x^2 + y^2} \]
        
                \vspace{0.5em}
        
                \textbf{Step 2:} Substitute the given value:
        
                \vspace{0.25em}
        
                \[ \sqrt{x^2 + y^2} = 3 \Rightarrow x^2 + y^2 = 9 \]
        
                \vspace{0.5em}
        
                \textbf{Answer:}
        
                \vspace{0.25em}
        
                \textcolor{darkgreen}{\( x^2 + y^2 = 9 \)}
        
            \end{adjustwidth}

            \textbf{Example 2:} Convert the equation \( r^2 + (z - 2)^2 = 4 \) to Cartesian coordinates.
            \vspace{0.5em}
        
            \begin{adjustwidth}{2em}{0pt}
        
                \textbf{Step 1:} Use the cylindrical to Cartesian identity:
        
                \vspace{0.25em}
        
                \[ r^2 = x^2 + y^2 \]
        
                \vspace{0.5em}
        
                \textbf{Step 2:} Substitute into the original equation:
        
                \vspace{0.25em}
        
                \[ x^2 + y^2 + (z - 2)^2 = 4 \]
        
                \vspace{0.5em}
        
                \textbf{Answer:}
        
                \vspace{0.25em}
        
                \textcolor{darkgreen}{\( x^2 + y^2 + (z - 2)^2 = 4 \)}
        
            \end{adjustwidth}

            \textbf{Example 3:} Convert the cylindrical equation \( r^2 + (z - 2)^2 = 4 \) into spherical coordinates.
            \vspace{0.5em}

            \begin{adjustwidth}{2em}{0pt}
            
                \textbf{Step 1:} Recall the cylindrical-to-spherical conversions:
                \vspace{0.25em}
            
                \[
                r = \rho\sin\phi, \quad z = \rho\cos\phi
                \]
            
                \vspace{0.5em}
            
                \textbf{Step 2:} Substitute into the equation:
                \vspace{0.25em}
            
                \[
                (\rho\sin\phi)^2 + (\rho\cos\phi - 2)^2 = 4
                \]
            
                \vspace{0.5em}
            
                \textbf{Step 3:} Simplify the expression:
                \vspace{0.25em}
            
                \[
                \rho^2\sin^2\phi + (\rho\cos\phi - 2)^2 = 4
                \]
            
                \vspace{0.5em}
            
                \textbf{Answer:}
                \vspace{0.25em}
            
                The equation in spherical coordinates is:  
                \textcolor{darkgreen}{\( \rho^2\sin^2\phi + (\rho\cos\phi - 2)^2 = 4 \)}
            
            \end{adjustwidth}
        
        \end{examplebox}

    \end{adjustwidth}


    % Subsection Title: Spherical Coordinates
    \subsection*{Spherical Coordinates}

    \begin{adjustwidth}{2em}{0pt}

        \begin{notebox}

            \textbf{Note:} \\
            Spherical coordinates describe a point in 3D space using three values:
                    
            \begin{itemize}
                \item \( \rho \): the distance from the origin
                \item \( \theta \): the angle from the positive x-axis in the xy-plane (same as polar coordinates)
                \item \( \phi \): the angle from the positive z-axis down to the point
            \end{itemize}

            This system is useful for problems with radial symmetry, like spheres and cones.

        \end{notebox}

        \vspace{0.5em}

        % Conversions
        \textbf{Spherical to Cartesian}

            \begin{itemize}
                \item \( x = \rho\sin\phi\cos\theta \)
                \item \( y = \rho\sin\phi\sin\theta \)
                \item \( z = \rho\cos\phi \)
            \end{itemize}

            \vspace{0.5em}

            \textbf{Cartesian to Spherical}

            \begin{itemize}
                \item \( \rho = \sqrt{x^2 + y^2 + z^2} \)
                \item \( \tan\theta = \frac{y}{x}  \rightarrow  \theta = \tan^{-1}\left(\frac{y}{x}\right) \)
                \item \( \phi = \arccos\left(\frac{z}{\sqrt{x^2 + y^2 + z^2}}\right) \)
            \end{itemize}

            \begin{examplebox}
        
                \textbf{Example 1:} Convert the Cartesian point \( \left(2,2,1\right) \) into Spherical coordinates.
                \vspace{0.5em}

                \begin{adjustwidth}{2em}{0pt}

                    \[ \rho = \sqrt{x^2 + y^2 + z^2} = \sqrt{2^2 + 2^2 + 1^2} = \sqrt{9} = 3 \]

                    \[ \theta = \tan^{-1}\left(\frac{y}{x}\right) = \tan^{-1}\left(\frac{2}{2}\right) = \tan^{-1}(1) = \frac{\pi}{4} \]

                    \[ \phi = \arccos\left(\frac{z}{\rho}\right) = \arccos\left(\frac{1}{3}\right) \]

                    \textbf{Answer:} \textcolor{darkgreen}{\( \left(3, \frac{\pi}{4}, \arccos\left(\frac{1}{3}\right)\right) \)}

                \end{adjustwidth}

                \break

                \textbf{Example 2:} Convert the Spherical point \( \left(4, \frac{\pi}{3}, \frac{\pi}{4}\right) \)
                \vspace{0.5em}

                \begin{adjustwidth}{2em}{0pt}

                    \[ x = \rho\sin\phi\cos\theta = 4 \cdot \sin\left(\frac{\pi}{4}\right) \cdot \cos\left(\frac{\pi}{3}\right) = 4 \cdot \frac{\sqrt{2}}{2} \cdot \frac{1}{2} = \sqrt{2} \]

                    \[ y = \rho\sin\phi\sin\theta = 4 \cdot \sin\left(\frac{\pi}{4}\right) \cdot \sin\left(\frac{\pi}{3}\right) = 4 \cdot \frac{\sqrt{2}}{2} \cdot \frac{\sqrt{3}}{2} = \sqrt{6} \]

                    \[ z = \rho\cos\phi = 4 \cdot \cos\left(\frac{\pi}{4}\right) = 4 \cdot \frac{\sqrt{2}}{2} = 2\sqrt{2} \]

                    \textbf{Answer:} \textcolor{darkgreen}{\( \left(\sqrt{2}, \sqrt{6}, 2\sqrt{2}\right) \)}

                \end{adjustwidth}

            \end{examplebox}

            \begin{examplebox}
        
                \subsection*{Converting Equations}
            
                \textbf{Example 1:} Convert the equation \( \rho = 5 \) to Cartesian coordinates.
                \vspace{0.5em}
            
                \begin{adjustwidth}{2em}{0pt}
            
                    \textbf{Step 1:} Spherical to Cartesian conversion:
            
                    \vspace{0.25em}
            
                    \[ \rho = \sqrt{x^2 + y^2 + z^2} \]
            
                    \vspace{0.5em}
            
                    \textbf{Step 2:} Substitute the given value:
            
                    \vspace{0.25em}
            
                    \[ \sqrt{x^2 + y^2 + z^2} = 5 \Rightarrow x^2 + y^2 + z^2 = 25 \]
            
                    \vspace{0.5em}
            
                    \textbf{Answer:}
            
                    \vspace{0.25em}
            
                    \textcolor{darkgreen}{\( x^2 + y^2 + z^2 = 25 \)}
            
                \end{adjustwidth}

                \vspace{0.5em}

                \textbf{Example 2:} Convert the spherical equation \( \rho = 2\sin\phi \) into both Cartesian and cylindrical coordinates.

                \vspace{0.5em}

                \begin{adjustwidth}{2em}{0pt}
                
                    \textbf{Step 1:} Use the identity for \( \rho \sin\phi = r \), where \( r = \sqrt{x^2 + y^2} \)
                
                    \vspace{0.25em}

                    Multiply both sides by \( \sin\phi \):
                
                    \[
                    \rho \sin\phi = 2\sin^2\phi
                    \]
                
                    \vspace{0.5em}
                
                    \textbf{Step 2:} Use substitution \( \rho \sin\phi = r \Rightarrow r = 2\sin^2\phi \)
                
                    \vspace{0.25em}
                
                    Now convert \( \sin^2\phi \) in terms of \( z \) and \( \rho \) using:
                    \[
                    z = \rho \cos\phi \Rightarrow \cos\phi = \frac{z}{\rho}, \quad \sin^2\phi = 1 - \cos^2\phi = 1 - \frac{z^2}{\rho^2}
                    \]
                
                    \vspace{0.5em}
                
                    \textbf{Step 3:} Substitute back:
                    \[
                    r = 2\left(1 - \frac{z^2}{\rho^2} \right)
                    \quad \text{and since} \quad \rho^2 = r^2 + z^2
                    \]

                    \break
                
                    Substituting \( \rho^2 \) and simplifying:
                    \[
                    r = 2\left(1 - \frac{z^2}{r^2 + z^2} \right)
                    = 2\left( \frac{r^2}{r^2 + z^2} \right)
                    \]
                
                    \[
                    \Rightarrow r(r^2 + z^2) = 2r^2
                    \Rightarrow r z^2 = r^2
                    \Rightarrow z^2 = r \quad \text{(in cylindrical)}
                    \]
                
                    \vspace{0.5em}
                
                    \textbf{Answer (Cylindrical):} \textcolor{darkgreen}{\( z^2 = r \)}
                
                    \vspace{0.5em}
                
                    \textbf{Step 4:} Convert to Cartesian:
                
                    Use \( r^2 = x^2 + y^2 \)
                
                    \[
                    z^2 = \sqrt{x^2 + y^2}
                    \]
                
                    \vspace{0.5em}
                
                    \textbf{Answer (Cartesian):} \textcolor{darkgreen}{\( z^2 = \sqrt{x^2 + y^2} \)}
                
                \end{adjustwidth}
            
            \end{examplebox}

    \end{adjustwidth}

\end{adjustwidth}

\section*{Tangent Planes}

\begin{adjustwidth}{2em}{0pt}



\end{adjustwidth}

\end{document}
