% document setup
\documentclass[10pt]{article}                               % 10pt font size
\usepackage[margin=1in]{geometry}                           % set the margin to 1 inch
\usepackage{mathpazo}                                       % use the font Palatino

\renewcommand{\familydefault}{\sfdefault}                   % set our font to default

% other packages
\usepackage{amsmath, amssymb}
\usepackage{changepage}
\usepackage[most]{tcolorbox}

% define colors
\definecolor{darkgreen}{RGB}{0,100,0}

\begin{document}

% Title: Calculus 3 Study Guide
\begin{center}
    {\Huge \textbf{Calculus 3 Study Guide}} \\
    \vspace{0.5em}
\end{center}

\vspace{1em}


% Section 1 | Coordinate Systems
\begin{tcolorbox}[breakable,
    enhanced,
    colback=white,
    colframe=black,
    boxrule=0.5pt,
    arc=2mm,
    left=1em,
    right=1em,
    top=1em,
    bottom=1em,
    overlay first={\draw[line width=.5pt] (frame.south west)--(frame.south east);},
    overlay middle={\draw[line width=.5pt] (frame.south west)--(frame.south east);\draw[line width=.5pt] (frame.north west)--(frame.north east);},
    overlay last={\draw[line width=.5pt] (frame.north west)--(frame.north east);}
    ]

    % Section Title: Coordinate Systems
    \section*{Coordinate Systems}


    \begin{adjustwidth}{2em}{0pt}

        % Subsection Title: Polar Coordinates
        \subsection*{Polar Coordinates}

        \begin{adjustwidth}{2em}{0pt}

            \begin{tcolorbox}[enhanced, colback=white, colframe=black, boxrule=0.5pt]

                \textbf{Note:} \\
                Polar coordinates describe a point in 2D space using two values:

                \begin{itemize}
                    \item \( r \): the distance from the origin
                    \item \( \theta \): the angle from the positive x-axis
                \end{itemize}

                This system is ideal for problems with circular or rotational symmetry, such as spirals or radial fields.

            \end{tcolorbox}


            \vspace{0.5em}

            % Conversions
            \textbf{Polar to Cartesian}

            \begin{itemize}
                \item \( x = r\cos\theta \)
                \item \( y = r\sin\theta \)
            \end{itemize}

            \vspace{0.5em}

            \textbf{Cartesian to Polar}

            \begin{itemize}
                \item \( r = \sqrt{x^2 + y^2} \)
                \item \( \tan\theta = \frac{x}{y}  \rightarrow  \theta = \tan^{-1}\left(\frac{x}{y}\right) \)
            \end{itemize}

        \end{adjustwidth}


        % Subsection Title: Cylindrical Coordinates
        \subsection*{Cylindrical Coordinates}

        \begin{adjustwidth}{2em}{0pt}

            \begin{tcolorbox}[enhanced, colback=white, colframe=black, boxrule=0.5pt]

                \textbf{Note:} \\
                Polar coordinates describe a point in 3D space using three values:

                \begin{itemize}
                    \item \( r \): The distance from the origin in the xy-plane
                    \item \( \theta \): The angle from the positive x-axis in the xy-plane (same as polar coordiantes)
                    \item \( z \): The height above (or below) the xy-plane
                \end{itemize}

                This system is useful for objects with circular symmetry around the z-axis, like cylinders and spirals.

            \end{tcolorbox}

            \vspace{0.5em}

            % Conversions
            \textbf{Cylindrical to Cartesian}

            \begin{itemize}
                \item \( x = r\cos\theta \)
                \item \( y = r\sin\theta \)
                \item \( z = z \)
            \end{itemize}

            \vspace{0.5em}

            \textbf{Cartesian to Cylindrical}

            \begin{itemize}
                \item \( r = \sqrt{x^2 + y^2} \)
                \item \( \tan\theta = \frac{x}{y}  \rightarrow  \theta = \tan^{-1}\left(\frac{x}{y}\right) \)
                \item \( z = z \)
            \end{itemize}

        \end{adjustwidth}


        % Subsection Title: Spherical Coordinates
        \subsection*{Spherical Coordinates}

        \begin{adjustwidth}{2em}{0pt}

            \begin{tcolorbox}[enhanced, colback=white, colframe=black, boxrule=0.5pt]

                \textbf{Note:} \\
                Spherical coordinates describe a point in 3D space using three values:
                        
                \begin{itemize}
                    \item \( \rho \): the ditance from the origin
                    \item \( \theta \): the angle from the positive x-axis in the xy-plane (same as polar coordiantes)
                    \item \( \phi \): the angle from the postive z-axis down to the point
                \end{itemize}

                This system is useful for problems with radial symmetry, like spheres and cones.

            \end{tcolorbox}

            \vspace{0.5em}

            % Conversions
            \textbf{Spherical to Cartesian}

                \begin{itemize}
                    \item \( x = \rho\sin\phi\cos\theta \)
                    \item \( y = \rho\sin\phi\sin\theta \)
                    \item \( z = \rho\cos\phi \)
                \end{itemize}

                \vspace{0.5em}

                \textbf{Cartesian to Spherical}

                \begin{itemize}
                    \item \( \rho = \sqrt{x^2 + y^2 + z^2} \)
                    \item \( \tan\theta = \frac{x}{y}  \rightarrow  \theta = \tan^{-1}\left(\frac{x}{y}\right) \)
                    \item \( \phi = \arccos\left(\frac{z}{\sqrt{x^2 + y^2 + z^2}}\right) \)
                \end{itemize}

        \end{adjustwidth}

    \end{adjustwidth}

\end{tcolorbox}

\end{document}
